\documentclass[11pt,a4paper]{article}

\usepackage{amsthm} %qed
\usepackage[cmex10]{amsmath}
\usepackage{amsfonts}
\usepackage{amssymb}
%\usepackage{graphicx}

\newcommand{\BIN}{\begin{bmatrix}}
\newcommand{\BOUT}{\end{bmatrix}}
\newcommand{\eq}[1]{eq.~(\ref{eq:#1})}
\newcommand{\rk}[1]{\mbox{rank}(#1)}

\DeclareMathOperator*{\minimize}{\min.}
\DeclareMathOperator*{\argmin}{\mbox{arg} \min.}
\newcommand{\st}{\mbox{s.t.}}
\newcommand{\half}{\frac{1}{2}}
\newcommand{\act}{\mathcal{A}}

\begin{document}
\date{January 22, 2021}
\author{Adrien Escande}

\title{Notes on the Goldfarb-Idnani paper}
\maketitle

These are notes on the dual QP solver paper of Goldfarb and Idnani~\cite{Goldfarb:mp:1983}.

\section{Problem statement and notations}
We consider the following optimization problem ($P$)
\begin{align}
	\minimize_x &\ \half x^T G x + a^T x \\
	\st &\ C^T x \geq b
\end{align}
and denote $f(x) = \half x^T G x + a^T x$ and $s(x) = C^T x - b$.
There are $m$ constraints and we note $\mathcal{K} = \left\{1 \ldots m\right\}$.

Conventions in the paper are such that the Lagrangian function is
\begin{equation}
	L(x,\lambda) = \half x^T G x + a^T x - \lambda^T(C^T x - b)
\end{equation}
where the Lagrange multipliers $\lambda$ needs to be positive.

Let's $\mathcal{A} \subseteq \mathcal{K}$ be a set of indices corresponding to active constraints, $N$ the corresponding columns of $C$, $b_\mathcal{A}$ the corresponding elements of $b$, and $u$ the corresponding elements of $\lambda$. The problem $P_\mathcal{A}$ is
\begin{align}
	\minimize_x &\ \half x^T G x + a^T x \\
	\st &\ N^T x = b_\mathcal{A}
\end{align}
If $x$ is solution to this problem\footnote{Assuming the constraints are linearly independent.} with $u\geq 0$, $(x,\act)$ is call a \emph{S-pair} (S for solution).

\section{Solving optimality conditions for $P_\act$}
The optimality conditions for $P_\act$ are
\begin{equation}
	\BIN G & -N \\ N^T & 0 \BOUT \BIN x \\ u \BOUT = \BIN -a \\ b_\act \BOUT \label{eq:optimPA}
\end{equation}
There are several ways to solve this system, including the nullspace approach and the Schur approach\cite[Chap. 16]{nocedal:book:2006}. Goldfarb and Idani are using the latter.

\subsection{A Schur complement approach}
Let's $M$ denote the matrix of the system. A simple application of block-inverse formula leads to
\begin{equation}
M^{-1} = \BIN G^{-1} - G^{-1} N (N^T G^{-1} N)^{-1} N^T G^{-1} & G^{-1} N (N^T G^{-1} N)^{-1} \\ - (N^T G^{-1} N)^{-1} N^T G^{-1} & (N^T G^{-1} N)^{-1} \BOUT
\end{equation}
Noting $S = (N^T G^{-1} N)^{-1}$, $N^* = S N^T G^{-1}$\newline and $H = G^{-1} - G^{-1} N S N^T G^{-1} = G^{-1}(I-N N^*)$, we have
\begin{equation}
M^{-1} = \BIN H & N^{*T} \\ - N^* & S \BOUT
\end{equation}
This leads to
\begin{equation}
	\BIN x \\ u \BOUT = \BIN -H a + N^{*T} b_\act \\ N^* a + S b_\act\BOUT \label{eq:primalDualSol}
\end{equation}

Note that because $N^* N = I$, we can also express $u$ as a function of $x$: $u = N^* (Gx+a) = N^* \nabla f$.

\subsection{Efficient implementation}
Following the paper, we consider the Cholesky decomposition
\begin{equation}
	G = L L^T
\end{equation}
and define $B = L^{-1} N$. The QR factorization of $B$ yields
\begin{equation}
	B = \BIN Q_1 & Q_2\BOUT \BIN R \\ 0 \BOUT
\end{equation}
We define $Q = \BIN Q_1 & Q_2\BOUT$ and
\begin{equation}
	J = \BIN J_1 & J_2\BOUT = L^{-T} \BIN Q_1 & Q_2\BOUT
\end{equation}
It follows that
\begin{align}
	S &= R^{-1}R^{-T} \\
	H &= J_2 J_2^T \\
	N^*& = R^{-1} J_1^T
\end{align}
If we define the following quantities:
\begin{align}
	\BIN \alpha_1 \\ \alpha_2 \BOUT &= \BIN J_1^T \\ J_2^T \BOUT a \\
	\beta &= R^{-T} b_\act \\
	\chi & = \BIN \beta \\ -\alpha_2 \BOUT
\end{align}
then \eqref{eq:primalDualSol} rewrites
\begin{equation}
	\BIN x \\ u \BOUT = \BIN J \chi \\ R^{-1} (\alpha_1 + \beta) \BOUT
\end{equation}
Finally, noting that $J^T G J = Q L^{-1} G L^{-T} Q = I$, we have that
\begin{equation}
	f = \frac{1}{2} x^T G x + x^T a = \frac{1}{2} \chi^T J^T G J \chi + \chi^T J^T a = \frac{1}{2} \chi^T \chi + \chi^T \alpha
\end{equation}

\section{Step computation}
Now consider that we have a S-pair $(x,\act)$, but that a constraint $p \in \mathcal{K} \backslash \act$ is violated. Noting $n^+$ the column of $C$ corresponding to this constraint and $b^+$ the associated element of $b$, we are looking for a step such that the optimality conditions are met again if we activate this constraint. The constraint was so far associated to a zero Lagrange multiplier. We want to solve
\begin{equation}
	\BIN G & -N & -n^+ \\ N^T & 0 & 0 \\ n^{+T} & 0 & 0\BOUT \BIN x + \delta x \\ u + \delta u \\ 0 + t\BOUT = \BIN -a \\ b_\act \\ b^+\BOUT
\end{equation}
Since $(x,u)$ solves \eq{optimPA}, we have that
\begin{equation}
	\BIN G & -N & -n^+ \\ N^T & 0 & 0 \\ n^{+T} & 0 & 0\BOUT \BIN \delta x \\ \delta u \\ t\BOUT = \BIN 0 \\ 0 \\ - s_p(x)\BOUT
\end{equation}

The first two equations yield
\begin{equation}
	\BIN \delta x \\ \delta u \BOUT = \BIN t H n^+ \\ -t N^* n^+ \BOUT
\end{equation}
and the paper notes $z = H n^+$ and $r = N^* n^+$.

The last equations $ n^{+T} \delta_x = s_p(x)$ them yields $t = -\dfrac{s_p(x)}{z^Tn^+}$. This full step $t$ might not be possible because some elements of $u - tr$ might become negative.

\section{Some results on matrices $H$ and $N^*$}
We note $N^+ = \BIN N & n^+\BOUT$, and $H^+$, $S^+$ the corresponding matrices.
Some tedious computations leads to
\begin{equation}
	S^+ = \BIN S & 0 \\ 0 & 0 \BOUT + \frac{1}{z^T n^+} \BIN -r \\ 1\BOUT \BIN -r \\ 1\BOUT^T
\end{equation}
and
\begin{equation}
	H^+ = H - \frac{z z^T}{z^T n^+}
\end{equation}

\bibliographystyle{unsrt}
\bibliography{bib}

\end{document}
